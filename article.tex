\documentclass{article}
\pdfsuppresswarningpagegroup=1

\author{Chinmaya Nagpal}
\title{May I Ask Where You Have Been, Photon?}
\date{18 November 2021}

\usepackage{svg,subfig,framed}
\usepackage[a4paper, total={6in, 8in}]{geometry}

\newcommand{\Q}{\bfseries Q: }
\newcommand{\A}{\par\textbf{A:} \normalfont}

\begin{document}

\maketitle

\begin{figure}[ht]
	\subfloat[Standard 
	setup]{\fbox{\includesvg[width=0.5\linewidth]{figures/double-slit.svg}}\label{fig:uncollimated}}
	\hspace*{\fill}
	\subfloat["Which-way" experiment
	]{\fbox{\includesvg[width=0.5\linewidth]{figures/collimated-double-slit.svg}}\label{fig:collimated}}
	\caption{Young's double slit interference experiment}
	\label{fig:double-slit}
\end{figure}

Quantum physics describes physical interactions that happen at the atomic and
sub-atomic levels, and underpins the foundation for most of modern technology.
However, physicists are still unsure how to use standard quantum theory to 
answer the following question: what path did a photon take through an 
interferometer before quantum measurement was performed?  A recent experiment 
conducted by A. Danan and colleagues at Tel Aviv University in Israel sheds 
light on how difficult it is to answer this question in any but the simplest of 
situations. 

In classical physics, a particle travelling between two points has a single
path, which can be calculated by specifying its state at either endpoint. On 
the other hand, in quantum physics, the trajectory a particle takes between two 
measurements is a weighted sum of every path it could have taken between the 
two measurements. This gives rise to the phenomenon of a particle interfering 
with itself before being measured, as demonstrated by experiments like the 
single-particle version of Young's double slit experiment. 

First performed in the early 19th century, the double experiment involves a 
relatively simple procedure: a beam of light is passed through two closely 
spaced slits and projected onto a screen, and the form of the light incident on 
the screen is observed. Because of the wave nature of light, different 
distances travelled by different wavefronts give rise to phase differences at 
the screen, causing an interference pattern to appear, as illustrated in 
figure~\ref{fig:uncollimated}. This appearance of light and dark bands is 
similar to how ripples interfere in a pond, creating alternating zones of 
maximal and minimal displacement based on how crests and troughs overlap.

In the modern version of this experiment, the continuous light source is 
replaced by a light source emitting individual particles, and a photosensitive 
plate is used as the screen. Over time, the same interference pattern appears 
on the photosensitive plate as in the classical experiment, indicating that 
individual particles undergo wave-like interference \textit{with themselves}, 
begging the question: 

\begin{framed}
\Q Where have you been, photon?  

\A Because the particle underwent wave-like interference with itself, it can be 
	said that it took a "superposition" of \textit{all possible paths} 
between the light source and the screen, meaning that it travelled through both 
the slits. \footnotemark \end{framed}

\footnotetext{Because the particle was not observed at any point between the 
source and the screen, another common position is to claim that \textit{one 
cannot say} which slit the photon has passed through. Either way, its path 
between source and detection is ambiguous}

\begin{figure}[h]
	\centering
	\includesvg[width=0.6\linewidth]{figures/interferometer1.svg}
	\caption{Simple Mach-Zehnder interferometer}
	\label{fig:interferometer1}
\end{figure}

One way to try to ascertain where each photon has been (in a so-called 
"which-way" experiment) is to insert a lens after each slit, as shown in 
figure~\ref{fig:collimated}.  Each lens only collimates the photons that arrive 
through the corresponding slit, causing the interference pattern to be replaced 
by two distinct spots. It would then seem safe to assume that a photon detected 
in the a spot on the screen must have travelled through the slit corresponding 
to that spot. It does not seem possible that a photon detected in, say, the 
upper spot would have gone through the lower lens. 


The Tel-Aviv experiment calls even this "common sense" assertion into question.  
The authors use the Mach-Zehnder interferometer, a device similar to Young's 
apparatus but better suited to precise theoretical treatment in the framework 
of quantum mechanics. It consists of a light source and a detector along with 
two beam splitters (basically half-silvered mirrors) and two mirrors, arranged 
as shown in figure~\ref{fig:interferometer1}. In the classical version of this 
experiment, the light source emits a continuous beam of light, which partly 
passes through and is partially reflected by the first beam splitter. The two 
beams of half the original intensity that emerge are then directed by the two 
mirrors to meet at the second beam splitter. Through a purely classical 
phenomenon, reflection at the vacuum-silver interface of a mirror causes a 
phase change to take place, causing the beams coming out of the second beam 
splitter to combine totally constructively on one side, and totally 
destructively on the other, in analogy to the bright and dark bands in the 
classical double slit experiment. 

Extending the analogy to the double-slit experiment, if the light source in the 
Mach-Zehnder interferometer is tuned to a low-enough intensity that single 
photons emitted can be resolved by the detector, it is found that the photon 
arrives at the detector 100\% of the time, demonstrating the phenomenon of a 
particle interfering with itself.  Of course, if we remove the second beam 
splitter, there is a 50\% chance that the photon will show up in our detector, 
and a 50\% chance that it will come out of the other branch of the 
interferometer. 


Again, we can answer titular question using our "common sense" approach:

\begin{framed}
\Q May I ask where you have been, photon?  

\A The photon's trajectory in the interferometer is a superposition of the 
	classical trajectories through the upper and lower arms of the 
	interferometer.
\end{framed}

\begin{figure}
	\centering
	\includesvg[width=0.7\linewidth]{figures/danan-1a.svg}
	\caption{Modified Mach-Zehnder interferometer}
	\label{fig:danan-1a}
\end{figure}

Danan et al. start off with a slightly modified version of the Mach-Zehnder 
interferometer, as shown in figure~\ref{fig:danan-1a}. To mark each possible 
path of the photon, they make each mirror oscillate vertically with a unique 
frequency.  Then, to make it possible to monitor the frequency with which the 
detected light oscillates, they use a stacked array of detectors in place of a 
simple light detector. As expected, when the experiment is performed with these 
simple alterations, the power spectrum shows equal contributions at frequencies 
$f_A$ and $f_B$ associated with mirrors $A$ and $B$. Whilst this does not 
directly indicated where individual photons have been---the photon source emits 
numerous photons, meaning that different photons could have caused the peaks at 
$f_A$ and $f_B$---it at least confirms that some of the photons have been at 
each mirror. 

In the second variation of the experiment, the same steps are repeated, but 
with the second beam splitter removed. This time, only the peak at $f_B$ 
remains, again confirming our "common sense" approach to particle trajectories. 

Next, a larger interferometer is built, as illustrated in figure 
\ref{fig:danan-2a}. There is a nested Mach-Zehnder interferometer in the upper 
arm, and the beam intensity is now split 1:2 between the upper and the lower 
arm, so the ratio of intensities at the vibrating mirrors $A$, $B$, and $C$ is 
1:1:1.  Yet again, the power spectrum shows what we would expect: peaks at all 
frequencies, with larger peaks at $f_E$ and $f_F$. 

The surprising result from this experiment is obtained when the following 
changes are made, as illustrated in figure \ref{fig:danan-2b}:
either the second beam splitter of the miniature Mach-Zehnder interferometer is 
flipped, or mirror $B$ is moved to slightly adjust the path length, causing 
destructive interference of the light propagating towards $F$. 

\begin{figure}
	\subfloat[]{\includesvg[width=0.5\linewidth]{figures/danan-2a.svg}\label{fig:danan-2a}}
	\subfloat[]{\includesvg[width=0.5\linewidth]{figures/danan-2b.svg}\label{fig:danan-2b}}
	\newline
	\hspace*{\fill}
	\subfloat[]{\includesvg[width=0.5\linewidth]{figures/danan-2c.svg}\label{fig:danan-2c}}
	\hspace*{\fill}
	\caption{Nested Mach-Zehnder interferometer}
	\label{fig:danan}
\end{figure}

\end{document}

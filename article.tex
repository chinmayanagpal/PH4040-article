\documentclass{article}
\pdfsuppresswarningpagegroup=1

\author{190015824}
\title{May I Ask Where You Have Been, Photon?}
\date{18 November 2021}

\usepackage{svg,subfig,framed,braket}
\usepackage[a4paper, total={6in, 8in}]{geometry}

\newcommand{\Q}{\bfseries Q: }
\newcommand{\A}{\par\textbf{A:} \normalfont}

\begin{document}

\maketitle

\begin{figure}[ht]
	\centering
	\subfloat[Standard 
	setup]{\fbox{\includesvg[width=0.5\linewidth]{figures/double-slit.svg}}\label{fig:uncollimated}}
	\hspace*{\fill}
	\subfloat[``Which-way" experiment
	]{\fbox{\includesvg[width=0.5\linewidth]{figures/collimated-double-slit.svg}}\label{fig:collimated}}
	\caption{Young's double slit interference experiment}
	\label{fig:double-slit}
\end{figure}

Quantum physics describes physical interactions that happen at the atomic and
sub-atomic levels, and underpins the foundation for most of modern technology.
However, physicists are still unsure how to use standard quantum theory to 
answer the following question: what path did a photon take through an 
interferometer before quantum measurement was performed? In a 2013 paper,
A. Danan and colleagues at Tel Aviv University in Israel shed light on how 
difficult it is to answer this question in the simplest of experimental 
situations, and highlight an alternative formalism in which their results find 
concise form\cite{danan}. 

In classical physics, a particle travelling between two points has a single
trajectory, which can be calculated by specifying its state at either endpoint.  
On the other hand, in quantum physics, performing a measurement on a particle 
at the beginning and the end of a period (so-called \textit{pre-} and 
\textit{post-selection}) only restricts the the trajectory a particle takes to 
a weighted sum of every path it could have 
taken\cite{aharonov2008}.\footnote{It is significant that this procedure is not 
possible in the classical world---there, measuring a particle's state at one 
endpoint gives us information about its whole trajectory} This gives rise to 
the phenomenon of a particle interfering with itself before being measured, as 
demonstrated by experiments like the single-particle version of Young's double 
slit experiment\cite{feynman}. 

First performed in the early 19th century, the double slit experiment involves 
a relatively simple procedure: a beam of light is passed through two closely 
spaced slits and projected onto a screen, and the form of the light incident on 
the screen is observed\cite{feynman}. Because of the wave nature of light, 
different distances travelled by different wavefronts give rise to phase 
differences at the screen, causing an interference pattern to appear, as 
illustrated in figure~\ref{fig:uncollimated}. This appearance of light and dark 
bands is similar to how ripples interfere in a pond, creating alternating zones 
of maximal and minimal displacement based on how the crests and troughs 
overlap.

In the modern version of this experiment, the continuous light source is 
replaced by a light source emitting individual particles, and a photosensitive 
plate is used as the screen. Over time, the same interference pattern appears 
on the photosensitive plate as in the classical experiment, indicating that 
individual particles undergo wave-like interference \textit{with themselves}, 
begging the question: 

\begin{framed}
\Q Where have you been, photon?  

\A Because the particle underwent wave-like interference with itself, it can be 
said that it took a \textit{superposition of all possible paths} between the 
light source and the screen, meaning that it travelled through \textit{both} 
slits at once\cite{lundeen}.  \footnotemark \end{framed}

\footnotetext{Because the particle was not observed at any point between the 
source and the screen, another common position is to claim that \textit{one 
cannot say} which slit the photon has passed through. Either way, its path 
between source and detection is ambiguous}

\begin{figure}[h]
	\centering
	\includesvg[width=0.6\linewidth]{figures/interferometer1.svg}
	\caption{Simple Mach-Zehnder interferometer}
	\label{fig:interferometer1}
\end{figure}

One way to try to ascertain where each photon has been (in a so-called 
``which-way" experiment) is to insert a lens after each slit, as shown in 
figure~\ref{fig:collimated}\cite{bohr,lundeen}. Each lens only collimates the 
photons that arrive through the corresponding slit, causing the interference 
pattern to be replaced by two distinct spots. It would then seem safe to assume 
that a photon detected in the a spot on the screen must have travelled through 
the slit corresponding to that spot. It does not seem possible that a photon 
detected in, say, the upper spot would have gone through the lower lens. 

The Tel-Aviv experiment calls even this ``common sense" reasoning into 
question\cite{lundeen,danan}. The authors use the Mach-Zehnder interferometer, 
a device similar to Young's apparatus but better suited to precise theoretical 
treatment in the framework of quantum mechanics\cite{lahiri}. It consists of a 
light source and a detector along with two beam splitters (basically 
half-silvered mirrors) and two mirrors, arranged as shown in 
figure~\ref{fig:interferometer1}. In the classical version of this experiment, 
the light source emits a continuous beam of light, which partly passes through 
and is partly reflected by the first beam splitter. The two beams (of half the 
original intensity) that emerge are then directed by the two mirrors to meet at 
the second beam splitter. Through a purely classical phenomenon, reflection at 
the vacuum-silver interface of a mirror causes the light to change phase (see 
labels on figure~\ref{fig:interferometer1}). A sequence of these phase changes 
causes the beams coming out of the second beam splitter to combine totally 
constructively on one side, and totally destructively on the other, analogous 
to the bright and dark bands in the classical double slit experiment. 

Extending this analogy, if the light source in the Mach-Zehnder interferometer 
is tuned to a low-enough intensity that single photons emitted can be resolved 
by the detector, it is found that the photon arrives at the detector 100\% of 
the time, demonstrating the phenomenon of a particle interfering with 
itself\cite{lahiri}. Of course, if we remove the second beam splitter, there is 
a 50\% chance that the photon will show up in our detector, and a 50\% chance 
that it will come out of the other branch of the interferometer. Again, we can 
attempt to answer the titular question using our ``common sense" approach: 

\begin{framed}

	\Q Where have you been, photon?  

	\A Like in the double-slit experiment, each photon undergoes wave-like 
	interference with itself. The trajectory of the photon between the 
	light source and the light detector is a superposition of a trajectory 
	through each arm of the interferometer. 

\end{framed}


\begin{figure}
	\centering
	\includesvg[width=0.7\linewidth]{figures/danan-1a.svg}
	\caption{Modified Mach-Zehnder interferometer. Graph taken from 
	\cite{danan}.}
	\label{fig:danan-1a}
\end{figure}

Danan et al. start off with a slightly modified version of the Mach-Zehnder 
interferometer, as shown in figure~\ref{fig:danan-1a}\cite{danan}. To mark each 
possible path of the photon, they make each mirror oscillate vertically with a 
unique frequency. Then, to make it possible to monitor the frequency with which 
the detected light oscillates (so called ``weak 
measurement"\cite{aharonov2008,danan}), they use a stacked array of detectors 
in place of a simple light detector. As expected, when the experiment is 
performed with these simple alterations, the power spectrum shows equal 
contributions at frequencies $f_A$ and $f_B$ associated with mirrors $A$ and 
$B$. Whilst this does not directly indicated where individual photons have 
been---the photon source emits numerous photons, meaning that different photons 
could have caused the peaks at $f_A$ and $f_B$---it at least confirms that some 
of the photons have been at each mirror\cite{danan}. 

In the second variation of the experiment, the same steps are repeated, but 
with the second beam splitter removed\cite{danan}. This time, only the peak at 
$f_B$ remains, again confirming our ``common sense" approach to particle 
trajectories. 

Next, a larger interferometer is built, as illustrated in figure 
\ref{fig:danan-2a}\cite{danan}. There is a nested Mach-Zehnder interferometer 
in the upper arm, and the beam intensity is now split 1:2 between the upper and 
the lower arm, so the ratio of intensities at the vibrating mirrors $A$, $B$, 
and $C$ is 1:1:1.  Yet again, the power spectrum shows what we would expect: 
peaks at all frequencies, with larger peaks at $f_E$ and $f_F$. 

 \begin{figure}
	\subfloat[]{\includesvg[width=0.5\linewidth]{figures/danan-2a.svg}\label{fig:danan-2a}}
	\subfloat[]{\includesvg[width=0.5\linewidth]{figures/danan-2b.svg}\label{fig:danan-2b}}
	\newline
	\hspace*{\fill}
	\subfloat[]{\includesvg[width=0.5\linewidth]{figures/danan-2c.svg}\label{fig:danan-2c}}
	\hspace*{\fill}
	 \caption{Nested Mach-Zehnder interferometer. Graphs taken from 
	 \cite{danan}.}
	\label{fig:danan}
\end{figure}


The surprising result from this experiment is obtained when the following 
changes are made, as illustrated in figure \ref{fig:danan-2b}:
either the second beam splitter of the miniature Mach-Zehnder interferometer is 
flipped, or mirror $B$ is moved to slightly adjust the path length, causing 
destructive interference of the light propagating towards $F$\cite{danan}.  
Seemingly, the only possible path the light could take to reach the detector is 
via mirror $C$. Yet, the observed spectrum contains frequency signatures from 
mirrors $A$ and $B$. To verify the destructive interference of light directed 
towards $F$, the path from $C$ to the detector is blocked 
(figure~\ref{fig:danan-2c}), yielding a null result. Somehow, light from the 
nested Mach-Zehnder interferometer is making its way to the detector!  
Furthermore, even though the photon somehow ``visits" mirrors $A$ and $B$, it 
avoids picking up the frequency signatures of mirrors $E$ and $F$! How? 

\begin{framed}

	\Q Where have you been, photon?  

	\A Now, it becomes hard to say. Using the same reasoning with which we 
	answered our question for the double slit experiment, it seems that the 
	light should be coming from the upper branch of the interferometer.  
	Consistent with this, upon blocking light emerging from mirror $C$, the 
	signal in the upper branch disappears. Yet, upon allowing light to 
	propagate through the upper branch, the frequencies $f_A$ and $f_B$ are 
	unexpectedly present in the power spectrum.

\end{framed}

There have been arguments against this experimental procedure, claiming that 
the vibration of the mirrors prevents full quantum interference from 
occuring\cite{comment,past}.  However, Danan et al. argue that a simpler 
explanation is possible, using description of quantum mechanics called the 
Two-State Vector Formalism (TSVF)\cite{danan,aharonov2008}. 

In the standard quantum mechanics formalism that undergraduate students of 
physics are familiar with, if a quantum system is measured (pre-selected) to be 
in a state $\ket \psi$ at time $t = t_1$, all information it is possible to 
know about the system at a subsequent time $t \geq t_1$ is contained in the 
state vector $$\ket{\hat U_{t t_1}\psi}$$ where $\hat U$ denotes its evolution 
in time after measurement at $t_1$, as given by the Schrödinger 
equation\cite{aharonov2008}.  Using this, one can express the quantum amplitude 
for observing a value $a_m$ of a physical observable $A$ at time $t$ as the 
inner product $$c_m = \braket{a_m|\hat U_{t t_1}\psi}$$ with the likelihood of 
this observation being given by ${|c_m|}^2$. 

Proponents of TSVF claim that the concept of a quantum state has the limitation 
that it is \textit{time-asymmetric}, as it is defined by measurements made in 
the \textit{past}\cite{aharonov,danan,past}. Of course, in classical physics, 
we are used to measuring a system at a time $t_1 < t$ to describe its state at 
$t$, but this does not amount to time-asymmetry. This is because for a 
classical system, the results of measurements performed in the future are 
already known after measurement performed in the past\cite{aharonov2008}. On 
the other hand, for a quantum system, measurement in the past only partially 
constrains the possible results of measurements at subsequent times, making the 
standard quantum mechanics formalism truly time-asymmetric. 

In Danan et al.'s interferometer, the position of the photon at the beginning 
of the experiment (i.e. at the light source) is known, as well as the position 
at the end (i.e.  at the light detector). That is to say, the system being 
described in the experiment is both pre- and post-selected. The authors claim 
that it is this aspect of the experiment that makes it hard to describe using 
the standard formalism and the associated ``common sense" approach to the past 
of a quantum particle.

Consider a simple quantum system that is pre-selected to be in the state
$\ket{\psi_1}$ at time $t_1$ and post-selected to be in the state 
$\ket{\psi_2}$ at time $t_2$. We can consider separately the quantum amplitudes 
of finding it in state $\ket{a_m}$ given that it is in state $\ket{\psi_1}$ at 
$t_1$, $$\braket{a_m|\hat U_{t t_1} \psi_1}$$ and of finding it in the state 
$\bra{\psi_2}$ at $t_2$ given that it is in the state $\ket{a_m}$ at time $t$, 
$$\braket{\psi_2|\hat U_{t t_2} a_m}$$ with $\hat U_{t_a t_b}$ denoting time 
evolution of a state between $t_a$ and $t_b$ as before\cite{aharonov2008}.

Using standard quantum mechanics, if we want to calculate the likelihood of 
observing the value $a_m$ for the observable $A$ at time $t$ between $t_1$ and 
$t_2$, we can multiply the individual complex amplitudes calculated above, with 
the former representing pre-selection, and the latter representing 
post-selection: $$c_m = \underbrace{\braket{\psi_2|\hat U_{t t_2} 
a_m}}_{post-selection} \underbrace{\braket{a_m|\hat U_{t t_1} 
\psi_1}}_{pre-selection}$$\cite{aharonov2008}. By performing a standard 
mathematical operation\footnote{called "transposing" an operator} under our 
usual quantum mechanical formalism, we can replace $\braket{\psi_2|\hat U_{t 
t_2} a_m}$ with the expression $\braket{\hat U_{t t_2}^{\dagger}\psi_2|a_m}$.  
This gives $$c_m = \braket{\hat U_{t t_2}^\dagger\psi_2| a_m} \braket{a_m|\hat 
U_{t t_1} \psi_1}$$. 

We can interpret this as involving the \textit{backwards-evolving} quantum 
state $\bra{\hat U_{t t_2}^{\dagger}\psi_2}$, ``prepared" at a time $t_2 > t$.  
TSVF treats the two-state vector $$\bra{\hat U_{t t_2}^{\dagger}\psi_2} 
~\ket{\hat U_{t t_1}\psi}$$ consisting of a backwards-evolving vector and a 
standard forward-evolving vector as containing absolute, concise information 
about a quantum state that has undergone pre and 
post-selection\cite{aharonov2008,danan}.

In context of the Tel-Aviv experiment, this means that it is the locations 
where the forwards and backwards-evolving paths coincide that will have their 
frequencies appear in the observed power spectrum\cite{danan}. Danan et al.  
argue that even though it is possible to explain their results using the 
standard quantum mechanical formalism\footnote{Using a phenomenon known as 
\textit{super-oscillations}}, it finds its simplest explanation using TSVF, 
where the past and the future are placed on equal footing.  


\bibliographystyle{unsrt} % We choose the "plain" reference style
\nocite{*}
\bibliography{refs} % Entries are in the refs.bib file

\end{document}

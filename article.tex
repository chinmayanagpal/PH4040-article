\documentclass{article}

\author{Chinmaya Nagpal}
\title{May I Ask Where You Have Been, Photon?}
\date{18 November 2021}


\begin{document}

\maketitle

Quantum physics describes physical interactions that happen at the atomic and
sub-atomic levels, and underpins the foundation for most of modern technology.
However, physicists are still unsure how to use standard quantum theory to 
answer the following question: what path did a photon take through an 
interferometer before quantum measurement was performed? 

In classical physics, a particle travelling between two points has a single
trajectory, which can be calculated by specifying its state at either endpoint
of the trajectory. On the other hand, in quantum physics, the trajectory a
particle takes between two measurements is a weighted sum, or superposition, of
every path it could have taken between the two measurements. This gives rise to
the phenomenon of a particle interfering with itself before being measured, as
demonstrated by experiments like the single-particle version of Young's double
slit experiment. 

Another apparatus that can demonstrate single-particle interference is the
Mach-Zehnder interferometer, which consists of a light source and a detector 
along with two beam splitters (basically half-silvered mirrors) and two 
mirrors, arranged as shown in [figure]. In the classical version of the 
experiment, the light source emits a continuous beam of light, which partly 
passes through the the first beam splitter, and is then reflected by the two 
mirrors to meet at the second beam splitter. Reflection at an interface between 
vacuum and the silvered side of a mirror causes a phase change through a purely 
classical phenomenon, which means that the beams coming out of the second beam 
splitter combine totally constructively on one side, and totally destructively 
on the other, in analogy to the bright and dark bands in the classical double 
slit experiment. Extending the analogy to the double-slit experiment, if the 
light source in the Mach-Zehnder interferometer is tuned to a low-enough 
intensity that single photons emitted can be resolved by the detector, it is 
found that the photon arrives at the detector 100\% of the time, demonstrating 
the phenomenon of a particle interfering with itself. Here, if we ask the 
photon where it has been, our "common sense" approach to quantum mechanics says 
that it has been in both the upper and the lower arm of the interferometer.

\end{document}
